%%%%%%%% ICML 2023 EXAMPLE LATEX SUBMISSION FILE %%%%%%%%%%%%%%%%%

\documentclass{article}

% Recommended, but optional, packages for figures and better typesetting:
\usepackage{microtype}
\usepackage{graphicx}
\usepackage{subfigure}
\usepackage{booktabs} % for professional tables

\usepackage{tikz}
% Corporate Design of the University of Tübingen
% Primary Colors
\definecolor{TUred}{RGB}{165,30,55}
\definecolor{TUgold}{RGB}{180,160,105}
\definecolor{TUdark}{RGB}{50,65,75}
\definecolor{TUgray}{RGB}{175,179,183}

% Secondary Colors
\definecolor{TUdarkblue}{RGB}{65,90,140}
\definecolor{TUblue}{RGB}{0,105,170}
\definecolor{TUlightblue}{RGB}{80,170,200}
\definecolor{TUlightgreen}{RGB}{130,185,160}
\definecolor{TUgreen}{RGB}{125,165,75}
\definecolor{TUdarkgreen}{RGB}{50,110,30}
\definecolor{TUocre}{RGB}{200,80,60}
\definecolor{TUviolet}{RGB}{175,110,150}
\definecolor{TUmauve}{RGB}{180,160,150}
\definecolor{TUbeige}{RGB}{215,180,105}
\definecolor{TUorange}{RGB}{210,150,0}
\definecolor{TUbrown}{RGB}{145,105,70}

% hyperref makes hyperlinks in the resulting PDF.
% If your build breaks (sometimes temporarily if a hyperlink spans a page)
% please comment out the following usepackage line and replace
% \usepackage{icml2023} with \usepackage[nohyperref]{icml2023} above.
\usepackage{hyperref}


% Attempt to make hyperref and algorithmic work together better:
\newcommand{\theHalgorithm}{\arabic{algorithm}}

\usepackage[accepted]{icml2023}

% For theorems and such
\usepackage{amsmath}
\usepackage{amssymb}
\usepackage{mathtools}
\usepackage{amsthm}

% if you use cleveref..
\usepackage[capitalize,noabbrev]{cleveref}

%%%%%%%%%%%%%%%%%%%%%%%%%%%%%%%%
% THEOREMS
%%%%%%%%%%%%%%%%%%%%%%%%%%%%%%%%
\theoremstyle{plain}
\newtheorem{theorem}{Theorem}[section]
\newtheorem{proposition}[theorem]{Proposition}
\newtheorem{lemma}[theorem]{Lemma}
\newtheorem{corollary}[theorem]{Corollary}
\theoremstyle{definition}
\newtheorem{definition}[theorem]{Definition}
\newtheorem{assumption}[theorem]{Assumption}
\theoremstyle{remark}
\newtheorem{remark}[theorem]{Remark}

% Todonotes is useful during development; simply uncomment the next line
%    and comment out the line below the next line to turn off comments
%\usepackage[disable,textsize=tiny]{todonotes}
\usepackage[textsize=tiny]{todonotes}


% The \icmltitle you define below is probably too long as a header.
% Therefore, a short form for the running title is supplied here:
\icmltitlerunning{Abstract Template for Data Literacy 2023/24}

\begin{document}

\twocolumn[
\icmltitle{Predicting most Efficient Interventions for Life Span Increase}

% It is OKAY to include author information, even for blind
% submissions: the style file will automatically remove it for you
% unless you've provided the [accepted] option to the icml2023
% package.

% List of affiliations: The first argument should be a (short)
% identifier you will use later to specify author affiliations
% Academic affiliations should list Department, University, City, Region, Country
% Industry affiliations should list Company, City, Region, Country

% You can specify symbols, otherwise they are numbered in order.
% Ideally, you should not use this facility. Affiliations will be numbered
% in order of appearance and this is the preferred way.
\icmlsetsymbol{equal}{*}

\begin{icmlauthorlist}
\icmlauthor{Daniel Flat}{equal,first}
\icmlauthor{Jackson Harmon}{equal,second}
\icmlauthor{Eric Nazarenus}{equal,third}
\icmlauthor{Aline Bittler}{equal,fourth}
\end{icmlauthorlist}

% fill in your matrikelnummer, email address, degree, for each group member
\icmlaffiliation{first}{Matrikelnummer 6604877, daniel.flat@student.uni-tuebingen.de, MSc Computer Science}
\icmlaffiliation{second}{Matrikelnummer 6770006, jackson.harmon@student.uni-tuebingen.de, MSc Machine Learning}
\icmlaffiliation{third}{Matrikelnummer 6703288, eric.nazarenus@student.uni-tuebingen.de, MSc Computer Science}
\icmlaffiliation{fourth}{Matrikelnummer 5379879, aline.bittler@student.uni-tuebingen.de, MSc Computer Science}

% You may provide any keywords that you
% find helpful for describing your paper; these are used to populate
% the "keywords" metadata in the PDF but will not be shown in the document
\icmlkeywords{Machine Learning, ICML}

\vskip 0.3in
]

% this must go after the closing bracket ] following \twocolumn[ ...

% This command actually creates the footnote in the first column
% listing the affiliations and the copyright notice.
% The command takes one argument, which is text to display at the start of the footnote.
% The \icmlEqualContribution command is standard text for equal contribution.
% Remove it (just {}) if you do not need this facility.

%\printAffiliationsAndNotice{}  % leave blank if no need to mention equal contribution
\printAffiliationsAndNotice{\icmlEqualContribution} % otherwise use the standard text.

\begin{abstract}
%Hennings explanation
%Put your abstract here. Mention, in two sentences, what you are planning to work on. 
%Then, mention which dataset you are planning to use. Include \href{https://noaadata.apps.nsidc.org/NOAA/G02135/south/daily/data/}{a link} to the dataset you are planning to use (If you are planning to collect your own data, explain how you are going to do so). If possible, mention (one sentence) how you came across this dataset, or why you decided to do this. Explain what kind of analysis you are planning. Finally, declare which results your are expecting to achieve. Your entire abstract should be at most 20 lines long.
%We will use mostly this same template for the project report, potentially with some minor updates.


%In this study we will use data from the on-going Robust Mouse Rejuvination (RMR) study to predict the effects of multiple intervention on the lifespan of rats. The goal is to be able to predict how each intervention contributes to the lifespan and then be able to predict the combination of interventions that will lead to the longest increase in lifespan using current data before the study concludes. Data will be collected from parsing the graph data located at (https://www.levf.org/projects/robust-mouse-rejuvenation-study-1/study-updates/november-5th-2023) into raw data, then fitting statistical models to each factor in the study. We also plan to synthesis this with data from other studies into one larger dataset. \\
%The motivation for this study is that solving the problem of aging is one of the large open questions currently faced in biology, which has been shown to be potentially tractable in recent years by reversing the aging process rather than just slowing it down. The RMR study is one of the largest on-going studies in this area. By being able to predict which factors are most likely to lead to the largest increase in longevity one could predict to which combinations should be studied further.

% v2 Daniel
In this study, we will use data from the ongoing
Robust Mouse Rejuvenation (RMR) study to
predict the effects of multiple interventions on
the lifespan of rats. Our goal is to
predict how each intervention contributes to
the lifespan and the combination of interventions that will lead to
the most prolonged increase in lifespan. 
%We will use the latest data of RMR before their study concludes.
We plan to collect their data by contacting the contributors of this study or by parsing the graph data located \href{https://www.levf.org/projects/robust-mouse-rejuvenation-study-1/study-updates/november-5th-2023}{here}. And then we will fit statistical models to each treatment factor. In addition, we plan to synthesize this with data from other studies into one larger dataset.
Our motivation is that solving the aging problem is one of the large questions currently facing biology. 
%This goal has been shown to be potentially tractable in recent years by reversing the aging process rather than only slowing it down. 
%The RMR study is one extensive study in that area that started in January 2023 \cite{levf2023}. 
By predicting which factors are most likely to lead to the most significant increase in longevity, one could predict which combinations should be studied further.
\end{abstract}

%v3 Draft Eric - Some reorganizing and ideas for phrasing. I think its good to start the abstract with the motivation. Probably need to shorten either way.

% The nature of aging appears to be one of the most enigmatic questions in the field of biology. A recent ongoing investigation, the Robust Mouse Rejuvenation study (RMR) signifies a potentially significant stride in our understanding of aging. The research not only seems to decelerate the aging process in rats but, intriguingly, suggests a reversal \cite{levf2023}.
% In our project we want to use data from the RMR study to
% predict the effects of multiple interventions on
% the lifespan of rats. Our goal is to
% predict how each intervention influences lifespan and identify combinations that yield the most substantial longevity increase. By forecasting which factors are pivotal for a noteworthy increase in lifespan, we aim to pinpoint combinations worthy of further exploration.
% To achieve this, we intend to utilize the latest data from the RMR study, procuring it before the study concludes. Our approach involves obtaining the data by either reaching out to the contributors of the study or extracting it from the graph data available \href{https://www.levf.org/projects/robust-mouse-rejuvenation-study-1/study-updates/november-5th-2023}{here}. Additionally, we plan to amalgamate this information with data from other studies such as \href{https://pubmed.ncbi.nlm.nih.gov/22587563/}{this}, creating a comprehensive dataset. Our methodology involves collecting raw data and subsequently applying statistical models tailored to analyze the specific characteristics of each treatment factor. This hopefully allows for a comprehensive understanding of the effects and intricacies associated with the interventions under consideration.
% \end{abstract}

\section{Contribution Plan}

%Explain here, in one sentence per person, what each group member is going to contribute. For example, you could write: Max Mustermann is going to collect and prepare data. Gabi Musterfrau and John Doe will run the data analysis. Jane Doe is responsible for visualizations. All authors will jointly write the text of the report. Note that you, as a group, a collectively responsible for the report. Your contributions should be roughly equal in amount and difficulty.

%v2 Daniel
Daniel and Aline are responsible for collecting and synthesizing the available data. Then Jackson and Eric will do the preprocessing, data analysis, and model fitting. Jackson will work on the introduction and motivation summaries. Furthermore, Eric and Daniel will create visualizations and figures. Finally, Daniel and Aline will analyze the limitations and biases of our study. 



\bibliography{bibliography}
\bibliographystyle{icml2023}

\end{document}


% This document was modified from the file originally made available by
% Pat Langley and Andrea Danyluk for ICML-2K. This version was created
% by Iain Murray in 2018, and modified by Alexandre Bouchard in
% 2019 and 2021 and by Csaba Szepesvari, Gang Niu and Sivan Sabato in 2022.
% Modified again in 2023 by Sivan Sabato and Jonathan Scarlett.
% Previous contributors include Dan Roy, Lise Getoor and Tobias
% Scheffer, which was slightly modified from the 2010 version by
% Thorsten Joachims & Johannes Fuernkranz, slightly modified from the
% 2009 version by Kiri Wagstaff and Sam Roweis's 2008 version, which is
% slightly modified from Prasad Tadepalli's 2007 version which is a
% lightly changed version of the previous year's version by Andrew
% Moore, which was in turn edited from those of Kristian Kersting and
% Codrina Lauth. Alex Smola contributed to the algorithmic style files.
