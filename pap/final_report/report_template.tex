%%%%%%%% ICML 2023 EXAMPLE LATEX SUBMISSION FILE %%%%%%%%%%%%%%%%%

\documentclass{article}

% Recommended, but optional, packages for figures and better typesetting:
\usepackage{microtype}
\usepackage{graphicx}
\usepackage{subfigure}
\usepackage{booktabs} % for professional tables

\usepackage{tikz}
% Corporate Design of the University of Tübingen
% Primary Colors
\definecolor{TUred}{RGB}{165,30,55}
\definecolor{TUgold}{RGB}{180,160,105}
\definecolor{TUdark}{RGB}{50,65,75}
\definecolor{TUgray}{RGB}{175,179,183}

% Secondary Colors
\definecolor{TUdarkblue}{RGB}{65,90,140}
\definecolor{TUblue}{RGB}{0,105,170}
\definecolor{TUlightblue}{RGB}{80,170,200}
\definecolor{TUlightgreen}{RGB}{130,185,160}
\definecolor{TUgreen}{RGB}{125,165,75}
\definecolor{TUdarkgreen}{RGB}{50,110,30}
\definecolor{TUocre}{RGB}{200,80,60}
\definecolor{TUviolet}{RGB}{175,110,150}
\definecolor{TUmauve}{RGB}{180,160,150}
\definecolor{TUbeige}{RGB}{215,180,105}
\definecolor{TUorange}{RGB}{210,150,0}
\definecolor{TUbrown}{RGB}{145,105,70}

% hyperref makes hyperlinks in the resulting PDF.
% If your build breaks (sometimes temporarily if a hyperlink spans a page)
% please comment out the following usepackage line and replace
% \usepackage{icml2023} with \usepackage[nohyperref]{icml2023} above.
\usepackage{hyperref}


% Attempt to make hyperref and algorithmic work together better:
\newcommand{\theHalgorithm}{\arabic{algorithm}}

\usepackage[accepted]{icml2023}

% For theorems and such
\usepackage{amsmath}
\usepackage{amssymb}
\usepackage{mathtools}
\usepackage{amsthm}

% if you use cleveref..
\usepackage[capitalize,noabbrev]{cleveref}

%%%%%%%%%%%%%%%%%%%%%%%%%%%%%%%%
% THEOREMS
%%%%%%%%%%%%%%%%%%%%%%%%%%%%%%%%
\theoremstyle{plain}
\newtheorem{theorem}{Theorem}[section]
\newtheorem{proposition}[theorem]{Proposition}
\newtheorem{lemma}[theorem]{Lemma}
\newtheorem{corollary}[theorem]{Corollary}
\theoremstyle{definition}
\newtheorem{definition}[theorem]{Definition}
\newtheorem{assumption}[theorem]{Assumption}
\theoremstyle{remark}
\newtheorem{remark}[theorem]{Remark}

% Todonotes is useful during development; simply uncomment the next line
%    and comment out the line below the next line to turn off comments
%\usepackage[disable,textsize=tiny]{todonotes}
\usepackage[textsize=tiny]{todonotes}

% Greek symbols in bibliography
\usepackage{textgreek}


% The \icmltitle you define below is probably too long as a header.
% Therefore, a short form for the running title is supplied here:
\icmltitlerunning{Project Report Template for Data Literacy 2023/24}

\begin{document}

\twocolumn[
\icmltitle{Predicting most Efficient Interventions for Life Span Increase}

% It is OKAY to include author information, even for blind
% submissions: the style file will automatically remove it for you
% unless you've provided the [accepted] option to the icml2023
% package.

% List of affiliations: The first argument should be a (short)
% identifier you will use later to specify author affiliations
% Academic affiliations should list Department, University, City, Region, Country
% Industry affiliations should list Company, City, Region, Country

% You can specify symbols, otherwise they are numbered in order.
% Ideally, you should not use this facility. Affiliations will be numbered
% in order of appearance and this is the preferred way.
\icmlsetsymbol{equal}{*}

\begin{icmlauthorlist}
\icmlauthor{Daniel Flat}{equal,first}
\icmlauthor{Jackson Harmon}{equal,second}
\icmlauthor{Eric Nazarenus}{equal,third}
\icmlauthor{Aline Bittler}{equal,fourth}
\end{icmlauthorlist}

% fill in your matrikelnummer, email address, degree, for each group member
\icmlaffiliation{first}{Matrikelnummer 6604877, daniel.flat@student.uni-tuebingen.de, MSc Computer Science}
\icmlaffiliation{second}{Matrikelnummer 6770006, jackson.harmon@student.uni-tuebingen.de, MSc Machine Learning}
\icmlaffiliation{third}{Matrikelnummer 6703288, eric.nazarenus@student.uni-tuebingen.de, MSc Computer Science}
\icmlaffiliation{fourth}{Matrikelnummer 5379879, aline.bittler@student.uni-tuebingen.de, MSc Computer Science}

% You may provide any keywords that you
% find helpful for describing your paper; these are used to populate
% the "keywords" metadata in the PDF but will not be shown in the document
\icmlkeywords{Machine Learning, ICML}

\vskip 0.3in
]

% this must go after the closing bracket ] following \twocolumn[ ...

% This command actually creates the footnote in the first column
% listing the affiliations and the copyright notice.
% The command takes one argument, which is text to display at the start of the footnote.
% The \icmlEqualContribution command is standard text for equal contribution.
% Remove it (just {}) if you do not need this facility.

%\printAffiliationsAndNotice{}  % leave blank if no need to mention equal contribution
\printAffiliationsAndNotice{\icmlEqualContribution} % otherwise use the standard text.

\begin{abstract}
Put your abstract here. Abstracts typically start with a sentence motivating why the subject is interesting. Then mention the data, methodology or methods you are working with, and describe results. 

In laboratoy studies it is widely common
\end{abstract}

\section{Introduction}\label{sec:intro}
Motivate the problem, situation or topic you decided to work on. Describe why it matters (is it of societal, economic, scientific value?). Outline the rest of the paper (use references, e.g.~to \Cref{sec:methods}: What kind of data you are working with, how you analyse it, and what kind of conclusion you reached. The point of the introduction is to make the reader want to read the rest of the paper.

\section{Data and Methods}\label{sec:methods}
In this section, describe \emph{what you did}. Roughly speaking, explain what data you worked with, how or from where it was collected, it's structure and size. Explain your analysis, and any specific choices you made in it. Depending on the nature of your project, you may focus more or less on certain aspects. If you collected data yourself, explain the collection process in detail. If you downloaded data from the net, show an exploratory analysis that builds intuition for the data, and shows that you know the data well. If you are doing a custom analysis, explain how it works and why it is the right choice. If you are using a standard tool, it may still help to briefly outline it. Cite relevant works. You can use the \verb|\citep| and \verb|\citet| commands for this purpose \citep{mackay2003information}.

% This is the template for a figure from the original ICML submission pack. In lecture 10 we will discuss plotting in detail.
% Refer to this lecture on how to include figures in this text.
% 
% \begin{figure}[ht]
% \vskip 0.2in
% \begin{center}
% \centerline{\includegraphics[width=\columnwidth]{icml_numpapers}}
% \caption{Historical locations and number of accepted papers for International
% Machine Learning Conferences (ICML 1993 -- ICML 2008) and International
% Workshops on Machine Learning (ML 1988 -- ML 1992). At the time this figure was
% produced, the number of accepted papers for ICML 2008 was unknown and instead
% estimated.}
% \label{icml-historical}
% \end{center}
% \vskip -0.2in
% \end{figure}

\citet{mccarthy2023mortality} investigated whether mortality is compressed and there is a maximum limit to the length of life, or postponed and there is no such limit. The study found that throughout history, there have been phases of both compression and postponement. The authors suggest that if there is a limit on the length of life, it has not yet been reached and predict that there will be more longevity records in the future. 

In contrast to the datasets we selected, \citet{fontana2010extending} had research on how dietary restrictions increase life expectancy and prevent illnesses such as diabetes, cancer and cardiovascular events. 



The \href{https://phenome.jax.org/projects/ITP1}{ITP1: Interventions Testing Program: Effects of various treatments on lifespan and related phenotypes in genetically heterogeneous mice (UM-HET3) (2004-2023)} includes treatments with 40 different substances. Some of these are used in combination or at different doses. Two treatments were dropped due to missing data. This leaves a total of 65 different treatments.

% Deleted data for PB125 (and Rapa_hi_start_stop) because of missing/wrong values


The \href{https://www.levf.org/projects/robust-mouse-rejuvenation-study-1/study-updates/november-5th-2023}{Robust Mouse Rejuvenation} study includes different combinations of four interventions. In total, there are ten different treatments. Unfortunately we did not get access to their dataset, but their evaluation plots are available online and we had to extract the data ourselves. 


\subsection{Interventions}
\begin{itemize}
    \item 17-a-estradiol (17aE2)
    \item 17-dimethylaminoethylamino-17-demethoxygeldanamycin hydrochloride (DMAG)
    \item 2-(2-Hydroxyphenyl)benzoxazole (HBX)
    \item 3-(3-hydroxybenzyl)-5-methylbenzo[d]oxazol-2(3H)-one (MIF098)
    \item 4-OH-a-phenyl-N-tert-butyl nitrone (4-OH-PBN)
    \item Acarbose (ACA) is an antidiabetic, which means it lowers the level of glucose in the blood, and can help improve weight control and the risk of cardiovascular events \cite{mccarty2015acarbose}. These events can be reduced by preventing the hyperglycaemic spikes after a meal \cite{mccarty2015acarbose}.
    \item aspirin (Asp)
    \item b-guanidinopropionic acid (bGPA)
    \item caffeic acid phenethyl ester (CAPE)
    \item canagliflozin (Cana)
    \item candesartan cilexetil (CC)
    \item captopril (Capt)
    \item curcumin (Cur)
    \item enalapril (Enal)
    \item fish oil (FO)
    \item geranylgeranyl acetone (GGA)
    \item glycine (Gly)
    \item green tea extract (GTE)
    \item INT-767 FXR/TGR5 agonist (INT-767)
    \item inulin (Inu)
    \item L-leucine (Leu)
    \item medium-chain triglyceride oil (MCTO)
    \item metformin (Met)
    \item metformin and rapamycin (MetRapa)
    \item methylene blue (MB)
    \item minocycline (Min)
    \item MitoQ (MitoQ)
    \item nicotinamide riboside (NR)
    \item nitroflurbiprofen (NFP)
    \item nordihydroguaiaretic acid (NDGA)
    \item oxaloacetic acid (OAA)
    \item Protandim (Prot)
    \item Rapamycin (Rapa) is an inhibitor of mTOR, the mammalian target of rapamycin. According to \citet{li2014rapamycin}, mTOR is a master regulator of cell growth and metabolism. It has been found that a defect in the mTOR signalling pathway is part of several diseases including cancer and diabetes \cite{li2014rapamycin}. Rapamycin also has an immunosuppressive effect \cite{saunders2001rapamycin}.
    \item rapamycin and acarbose (RaAc)
    \item resveratrol (Res)
    \item (R/S)-1,3-butanediol (BD)
    \item simvastatin (Sim)
    \item sulindac (Sul)
    \item syringaresinol (Syr)
    \item TM5441 (TM5441)
    \item ursodeoxycholic acid (UDCA)
    \item ursolic acid (UA)
    
    \item Haematopoietic stem cells (HSCs): Haematopoietic stem cells are the cells responsible for making blood cells. These cells have the ability to make all types of blood cells. These include white blood cells, red blood cells and platelets. HSCs from young mice could lead to life extension in old mice. This is shown by a number of studies \cite{levf_update1}\cite{robust_mouse_study_1}.
    
    \item Galactose-conjugated Navitovclax (Gal-Nav): Navitoclax is a senolytic that reduces the number of senescent cells. These are cells that no longer divide, which is a protection against cancer \cite{collado2007cellular}. Treatment with senolytics can, for example, save nerve cells \cite{wagner2023aging}. Reducing senescent cells could reduce cancer and immune decline \cite{robust_mouse_study_1}. Navitoclax is toxic to platelets, so they have conjugated it with the sugar galactose. This conjugate is inactive and can be activated by galactosidase (in senescent cells) \cite{levf_update1}.
    
    \item Telomerase Gene Therapy (mTERT): The telomeres of the chromosomes shorten with each cell division until they can no longer divide and the cell becomes senescent. Telomerase, which is not inactive in most normal cells, causes the cells to divide again. \citet{bernardes2012telomerase}Bernardes De Jesus et al. state that a characteristic of human cancer cells is the overexpression of telomerase reverse transcriptase (TERT). Increased TERT expression in mice genetically modified for cancer resistance effectively reduces age-related telomere damage, leading to a delay in the ageing process. In addition, the accelerated ageing caused by telomere shortening can be partially alleviated by reactivating telomerase \cite{bernardes2012telomerase}. 
\end{itemize}

\subsection{Gompertz Law of Mortality}
The Gompertz law of mortality \cite{gompertz} is a mathematical model that describes the mortality rate of populations as they age. The Gompertz law is expressed by the following equation:

\[ \mu(x) = \alpha \cdot e^{\beta x} \]
\begin{enumerate}
\item  \( \mu(x) \): The force of mortality at age \( x \), which is the instantaneous rate of death for individuals at age \( x \).

\item \( \alpha \): The baseline mortality rate or the initial level of mortality at a given age. It reflects the risk of death at the start of the aging process.

\item \( \beta \): The Gompertz coefficient, influencing the rate of increase in mortality with age. It characterizes the exponential growth in mortality with advancing age.
\end{enumerate}

The Gompertz law is particularly useful for modeling mortality rates in adult populations, as it captures the increasing risk of death that comes with age.


\section{Results}\label{sec:results}
In this section outline your results. At this point, you are just stating the outcome of your analysis. You can highlight important aspects (``we observe a significantly higher value of $x$ over $y$''), but leave interpretation and opinion to the next section. This section absoultely \emph{has} to include at least two figures.

\section{Discussion \& Conclusion}\label{sec:conclusion}
Use this section to briefly summarize the entire text. Highlight limitations and problems, but also make clear statements where they are possible and supported by the analysis. 

\section*{Contribution Statement}
%Explain here, in one sentence per person, what each group member contributed. For example, you could write: Max Mustermann collected and prepared data. Gabi Musterfrau and John Doe performed the data analysis. Jane Doe produced visualizations. All authors will jointly wrote the text of the report. Note that you, as a group, a collectively responsible for the report. Your contributions should be roughly equal in amount and difficulty.

Daniel and Aline are responsible for collecting and synthesizing the available data. Then Jackson and Eric will do the preprocessing, data analysis, and model fitting. Jackson will work on the introduction and motivation summaries. Furthermore, Eric and Daniel will create visualizations and figures. Finally, Daniel and Aline will analyze the limitations and biases of our study. 
\todo{adjust}

\section*{Notes} 
Your entire report has a \textbf{hard page limit of 4 pages} excluding references. (I.e. any pages beyond page 4 must only contain references). Appendices are \emph{not} possible. But you can put additional material, like interactive visualizations or videos, on a githunb repo (use \href{https://github.com/pnkraemer/tueplots}{links} in your pdf to refer to them). Each report has to contain \textbf{at least three plots or visualizations}, and \textbf{cite at least two references}. More details about how to prepare the report, inclucing how to produce plots, cite correctly, and how to ideally structure your github repo, will be discussed in the lecture, where a rubric for the evaluation will also be provided.


\bibliography{bibliography}
\bibliographystyle{icml2023}

\end{document}


% This document was modified from the file originally made available by
% Pat Langley and Andrea Danyluk for ICML-2K. This version was created
% by Iain Murray in 2018, and modified by Alexandre Bouchard in
% 2019 and 2021 and by Csaba Szepesvari, Gang Niu and Sivan Sabato in 2022.
% Modified again in 2023 by Sivan Sabato and Jonathan Scarlett.
% Previous contributors include Dan Roy, Lise Getoor and Tobias
% Scheffer, which was slightly modified from the 2010 version by
% Thorsten Joachims & Johannes Fuernkranz, slightly modified from the
% 2009 version by Kiri Wagstaff and Sam Roweis's 2008 version, which is
% slightly modified from Prasad Tadepalli's 2007 version which is a
% lightly changed version of the previous year's version by Andrew
% Moore, which was in turn edited from those of Kristian Kersting and
% Codrina Lauth. Alex Smola contributed to the algorithmic style files.
